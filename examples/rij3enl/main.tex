\documentclass{uclamsc}

\includeonly{
chapters/dedicatoria, 
chapters/agradecimientos, 
chapters/introduccion, 
chapters/el-problema, 
chapters/marco-teorico, 
chapters/marco-metodologico, 
chapters/resultados,
chapters/discusion,
chapters/conclusiones,
chapters/recomendaciones,
chapters/anexos
}

\final{false}
\preambulo
\loadglsentries{chapters/glosario.tex}

\begin{document}
		\titulo{Robot Inteligente para Jugar Tres en Línea}
		\autor{Juan Rada Vilela}
		\citarcomo{Rada} %para habilitar comando \yo, y usarlo en \fuentecuadro{\yo}
		\decanato{Ciencias y Tecnología}
		\postgrado{Ciencias de la Computación}
		\ciudad{Cabudare}
		\diadefensa{03}
		\mesdefensa{Mayo}
		\annodefensa{2005}
		\tutor{Douglas Domínguez}
		\primerjurado{Jesús Contreras}
		\segundojurado{Luis Alvarado}
		\resumen{
		Este trabajo de grado utiliza la Teoría de Juegos para crear Inteligencia Artificial y así lograr que un robot tome óptimas decisiones en situaciones finitas, no cooperativas y de información perfecta. El campo de aplicación está enfocado hacia el juego Tres en Línea, donde el robot participa como jugador y utiliza la estrategia definida por el Equilibrio Nash para tomar óptimas decisiones en el juego. Aunque Tres en Línea es sólo un ejemplo de situación, se explica detalladamente el procedimiento para conseguir un Equilibrio Nash en cualquier situación finita de información perfecta; el cual determina una n-tupla de estrategias tal que, cuando los n jugadores utilizan su estrategia respectiva Sn, todos obtendrán el mejor beneficio posible. En otras palabras, ningún jugador puede obtener mejor beneficio que el obtenido con la estrategia Sn. Las estrategias que conforman el Equilibrio Nash para este juego son no perdedoras, es decir, el robot jamás perderá un juego. Ahora, para la construcción del robot se utiliza el equilibrio estático de cuerpos rígidos; lo que permite siempre un equilibrio en la estructura y así los actuadores ejecutan un trabajo mínimo. El robot consiste en tres articulaciones y cuatro grados de libertad para posicionar la muñeca en cualquier punto del campo de juego. Los grados de libertad son transversal vertical, transversal radial, transversal rotacional y torsión. Su efector final es un marcador que a través de un sistema mecánico es capaz de dibujar el símbolo O (correspondiente a la jugada del robot).}
	\palabrasclave{Robótica, Inteligencia Artificial, Teoría de Juegos, Equilibrio Nash, Algoritmo de Zermelo, Microcontrolador 8951}

	\title{Intelligent Robot for Playing Tic-Tac-Toe}
	\abstract{This is the abstract}
	\keywords{These are the keywords}
	
	\habilitarpendientes	
	\habilitarnotas
	
	\configurar
\begin{preliminares}
	\nota{hay algunas referencias duplicadas, pero por falta de tiempo no sé cuales son. si lo saben avisen :)}
	\hacercaratula
	\hacerpresentacion
	\haceraprobacion
	%!TEX root = ../trabajo.tex
\preliminar{Dedicatoria}


	%!TEX root = ../main.tex
\preliminar{Agradecimientos}

	\hacerindice
	\hacerresumen
	\hacerabstract
\end{preliminares}

\begin{contenido}
	\introduccion

%Manual de comandos del documentclass uclamsc, para los trabajos de grado de la Universidad Centroccidental Lisandro Alvarado (UCLA), Venezuela.

\noindent
Página del proyecto: \url{https://code.google.com/r/melfeverafter-uclamsc/}. Esta versión es derivada de: \url{https://code.google.com/p/latex-uclamsc/}

El código fuente para la creación de este documento (\texttt{manual.pdf}), se encuentra en el directorio del proyecto \texttt{ejemplos/manual}.

En el directorio \texttt{ejemplos/msctesis} se encuentra un cascarón vacío para el trabajo de grado, que se puede usar como punto de partida.


El cuerpo básico para el documento tiene la siguiente forma:

\lstset{commentstyle=\color[gray]{0.3}\itshape, escapeinside={(*}{*)}}

\begin{listadoindexado}{Cuerpo del Documento}{lst:body}{[LaTeX]TeX}
\documentclass{uclamsc}
\includeonly{
	% Capítulos a incluir en la compilación
}
\bibliografia{(*\textit{nombre del archivo principal}*)}
% \habilitarpendientes	
% \habilitarnotas
% Comandos de indentificación
\begin{document}
	% Incluir resumen \resumen{}
	% Incluir abstract \abstract{}
	\begin{preliminares}
		% Comandos de preliminares
	\end{preliminares}
	\begin{contenido}
		% Comandos de contenido
	\end{contenido}
	\hacerbibliografia
	\begin{anexos}
		% Comandos de anexos
	\end{anexos}
\end{document}
\end{listadoindexado}
	%!TEX root = ../main.tex
\capitulo{El Problema}

\seccion{Planteamiento del Problema}

 T. Kanade (2004), en entrevista con Popular Mechanics en Español opina que 
\textquotedblleft la mejor definición de robot es aquella que habla de un sistema que registra el 
mundo real exterior para interpretarlo y tomar decisiones útiles e inteligentes que 
tengan un impacto en el mundo real\textquotedblright. A lo que Popular Mechanics en Español (2004) 
considera que \textquotedblleft Es, básicamente, a lo que nos dedicamos los humanos. A decidir.\textquotedblright 	\pendiente{Revisar redaccion}
	
 Muchas situaciones de la vida cotidiana se basa en la toma de decisiones: una 
pareja decide si ir al cine o a un concierto, una empresa decide si entrar ó no a un 
mercado. Estas son decisiones que se toman de acuerdo a intereses basándose en 
instintos, experiencias o incluso en sentido común. En nada difieren de los juegos: en 
ajedrez, dos personas deciden sobre cómo jugar para lograr un jaque mate; en Tres en 
Línea dos personas deciden sobre donde colocar los símbolos X y O para lograr tres 
en línea. 

 La toma de decisiones en robótica requiere de inteligencia artificial, la cual 
posee diversas herramientas para su modelaje. Tales herramientas son: las redes 
neuronales y sus algoritmos de aprendizaje iterativo; los sistemas inteligentes y sus 
inferencias lógicas sobre la base de conocimientos; los algoritmos genéticos y su 
evolución por la teoría darwiniana; entre otras. 

 Aunque con estas herramientas cualquier robot puede llegar a tomar las 
mejores decisiones, el problema radica en extensas situaciones finitas de información 
perfecta con interdependencia estratégica (e.g. Tres en Línea) donde no es fácil 
demostrar que el resultado obtenido es el óptimo. 

 Por ejemplo, se construye un robot capaz de tomar decisiones en el juego Tres 
en Línea y su inteligencia artificial está definida por un sistema inteligente. De esta 
forma, el robot jugará utilizando la experiencia previa y con el tiempo jugará mejor. 
El resultado es empírico, aunque tenga mucha experiencia surge la pregunta: se han 
tomado las mejores decisiones? 

 Utilizando el mismo ejemplo, pero en lugar de utilizar un sistema inteligente 
su inteligencia artificial está definida por una red neuronal. Sin entrar en detalles 
respecto a su diseño e independientemente si son de aprendizaje supervisado o no, el 
robot aprenderá a tomar decisiones. Si toma una decisión errónea, se modifican los 
pesos de las neuronas y, de esta forma, convergerá hacia una buena toma de 
decisiones; pero aunque el error sea mínimo, surge nuevamente la pregunta: se han 
tomado las mejores decisiones? 

 La misma pregunta surge utilizando algoritmos genéticos.  

 Una forma de responder a esta pregunta es analizando todas las posibles 
combinaciones estratégicas del juego para determinar si en realidad se tomaron las 
mejores decisiones, lo cual es poco práctico debido al tamaño del árbol de juego. 
 
Asimismo, esta inquietud no sólo tiene lugar en el juego Tres en Línea, existirá 
siempre que se utilicen estas herramientas para la toma de decisiones en situaciones 
similares. 

  Por el problema expuesto anteriormente y en búsqueda de una herramienta 
que permita tomar decisiones óptimas y justificadas, se plantea el uso de la teoría de 
juegos para crear Inteligencia Artificial y así solventar (de la mejor manera posible) 
cualquier problema de decisión en estas situaciones. Luego, para su aplicación en el 
campo de la robótica, se construirá un robot que utilice esta herramienta para tomar 
decisiones en el juego Tres en Línea.  

\seccion{Objetivos}
	\subseccion{Objetivos Generales}
		Construir un robot inteligente capaz de jugar Tres en Línea. 
		  Por el problema expuesto anteriormente y en búsqueda de una herramienta 
		que permita tomar decisiones óptimas y justificadas, se plantea el uso de la teoría de 
		juegos para crear Inteligencia Artificial y así solventar (de la mejor manera posible) 
		cualquier problema de decisión en estas situaciones. Luego, para su aplicación en el 
		campo de la robótica, se construirá un robot que utilice esta herramienta para tomar 
		decisiones en el juego Tres en Línea.  
		
	\subseccion{Objetivos Específicos}
	\begin{enumeracion}
		\item Diagnosticar la utilidad de la teoría de juegos en problemas para la toma de decisiones en el campo de la robótica. 
		\item Estudiar la factibilidad técnica, económica y operativa del sistema propuesto. 
		\item Construir un robot inteligente capaz de jugar Tres en Línea. 
	\end{enumeracion}

\seccion{Justificación e Importancia}
Es claro que la toma de decisiones es importante para cualquier ser vivo, pero 
más importante aún es tomar una óptima decisión. Este resultado se logra utilizando 
la teoría de juegos que, con fundamentos matemáticos, demuestra que las decisiones 
tomadas son óptimas. 

 Con este trabajo de investigación se pretende introducir la teoría de juegos en 
el campo de la robótica con la construcción de un robot inteligente capaz de tomar 
decisiones óptimas en el juego Tres en Línea. Aunque este juego es sólo un ejemplo 
de situación donde se toman decisiones, se explicará detalladamente la teoría para su 
aplicación a cualquier situación finita de información perfecta.  

 Vale destacar que este trabajo es de carácter innovador dentro de la 
Universidad Fermín Toro, ya que hasta la fecha (Abril de 2005) no se han realizado 
trabajos de grado donde se utiliza la teoría de juegos como herramienta para la toma 
de decisiones a nivel de inteligencia artificial.  

 Una opinión sobre el uso de la teoría de juegos en esta área es presentada por 
J. Roach (2004): \textquotedblleft Los objetivos de estas dos áreas son tan similares que es solamente 
natural, para la teoría de juegos e inteligencia artificial, crear una sinergia para 
formular novedosos enfoques en la solución de diversos problemas\textquotedblright. 
 
Por otro lado, el robot construido será adquirido por la Fundación para la 
Ciencia y Tecnología del Estado Lara (FUNDACITE Lara) para fomentar 
investigaciones en el campo de la robótica. Por este motivo, se creará un robot 
flexible para que pueda ser utilizado en una mayor gama de aplicaciones. 

Este trabajo de grado está orientado al beneficio de la comunidad por las 
siguientes razones 
\begin{enumeracion}
	\item Se explica detalladamente una herramienta que permite tomar decisiones óptimas, 
	la cual podrá ser utilizada para desarrollar desde sistemas hasta robots inteligentes 
	que no sólo tomarán buenas decisiones sino que tomarán las mejores. 
	\item  Se explica el diseño del robot desde un punto de vista mecánico, lo cual permitirá 
	servir de apoyo bibliográfico para las consideraciones que se deben tomar en 
	cuenta al momento de la construcción de cualquier robot. 
	\item El código del software desarrollado en alto y bajo nivel están comentados y bajo 
	el esquema de software libre, lo cual dará una mejor visión sobre cómo crear 
	Inteligencia Artificial utilizando la teoría de juegos además de cómo controlar un 
	robot; de igual manera se podrá disfrutar de los beneficios que ofrece el software 
	libre. 
\end{enumeracion}

 Por lo antes mencionado, este trabajo de grado se ubica en el Polo 2 (Hombre, 
Ciudad y Territorio) que al intersectarlo con el eje correspondiente, diseño y 
mantenimiento de sistemas inteligentes, ubica la línea de investigación Inteligencia 
Artificial y Robótica. 

\seccion{Alcances y Limitaciones}
	\subseccion{Alcances}
	\begin{enumeracion}
		\item Construir un brazo robótico capaz de tomar decisiones óptimas en el juego Tres 
		en Línea. 
		\item Construir un efector final capaz de dibujar la jugada del robot en el campo de 
		juego. 
		\item Conseguir una estrategia óptima para cada jugador.
		\item Interpretar el campo de juego por visión artificial.
		\item Facilitar la interacción humano – robot mediante un tablero de juego.
		\item Integrar el software y el hardware mediante un protocolo de comunicación 
		bidireccional. 
	\end{enumeracion}
	
	\subseccion{Limitaciones}
		\begin{enumeracion}
			\item El robot sólo dibujará su jugada respectiva utilizando el símbolo O. 
			\item Las jugadas realizadas en el campo de juego deberán realizarse dentro de los 
			márgenes establecidos y de un tamaño no menor a 4 cm2.  
			\item Los algoritmos de visión artificial están diseñados para reconocer los símbolos O, 
			cualquier otro símbolo será interpretado como una jugada del adversario. 
			\item  La posición de la cámara digital no podrá alterarse luego de reiniciar el 
			microcontrolador. 
			\item La iluminación del escenario no podrá alterarse. 
			\item El campo de juego deberá permanecer en la posición establecida. 
			\item El robot no evita obstáculos. 
		\end{enumeracion}
	
	%!TEX root = ../trabajo.tex
\capitulo{Marco Teórico}

\seccion{Antecedentes de la Investigación}

\seccion{Bases Teóricas}

\seccion{Bases Legales}

\seccion{Definición de Términos Básicos}

%Si se va a usar el glosario
%\hacerglosario

\seccion{Sistema de Hipótesis}

\seccion{Operacionalización de las Variables}
	%!TEX root = ../main.tex
\capitulo{Marco Metodológico}

\seccion{Tipo de Investigación}
Esto es experimental según~\citar{laumanns2001:effects}.\\

\begin{cuadro}{l rrrrrrr}{Título del cuadro}{etiqueta}
	\toprule
	Audio Name&\multicolumn{7}{c}{Sum of Extracted Bits} \\ [0.5ex]    
	\midrule
	Police   & 5 & -1 &  5& 5& -7& -5& 3\\  % Entering row contents 
	Midnight & 7 & -3 &  5& 3& -1& -3& 5\\ 
	News     & 9 & -3 &  7& 9& -5& -1& 9\\[1ex] % [1ex] adds vertical space 
	\bottomrule
	\fuentecuadro{8}{El autor}

\end{cuadro}

\seccion{Población y Muestra}
En el cuadro~\ref{etiqueta} se puede observar que sirvió la etiqueta

\seccion{Diseño de la Investigación o Procedimiento}


\seccion{Técnicas e Instrumentos de Recolección de Datos}

\seccion{Técnicas de Procesamiento y Análisis de los Datos}
	%%
% Copyright (c) {año} {nombre} <{email}>.
%
% This program is free software: you can redistribute it and/or modify
% it under the terms of the GNU General Public License as published by
% the Free Software Foundation, either version 3 of the License, or
% (at your option) any later version.
%
% This program is distributed in the hope that it will be useful,
% but WITHOUT ANY WARRANTY; without even the implied warranty of
% MERCHANTABILITY or FITNESS FOR A PARTICULAR PURPOSE.  See the
% GNU General Public License for more details.
%
% You should have received a copy of the GNU General Public License
% along with this program.  If not, see <http://www.gnu.org/licenses/>.
%%

\capitulo{Resultados}

	%!TEX root = ../trabajo.tex

%%
% Copyright (c) {año} {nombre} <{email}>.
% 
% This program is free software: you can redistribute it and/or modify
% it under the terms of the GNU General Public License as published by
% the Free Software Foundation, either version 3 of the License, or
% (at your option) any later version.
% 
% This program is distributed in the hope that it will be useful,
% but WITHOUT ANY WARRANTY; without even the implied warranty of
% MERCHANTABILITY or FITNESS FOR A PARTICULAR PURPOSE.  See the
% GNU General Public License for more details.
% 
% You should have received a copy of the GNU General Public License
% along with this program.  If not, see <http://www.gnu.org/licenses/>.
%%

\capitulo{Conclusiones}

	%!TEX root = ../trabajo.tex
\capitulo{Recomendaciones}
\end{contenido}

\hacerbibliografia{main}

\begin{anexos}
	\haceranexos
	%!TEX root = ../trabajo.tex

%%
% Copyright (c) {año} {nombre} <{email}>.
% 
% This program is free software: you can redistribute it and/or modify
% it under the terms of the GNU General Public License as published by
% the Free Software Foundation, either version 3 of the License, or
% (at your option) any later version.
% 
% This program is distributed in the hope that it will be useful,
% but WITHOUT ANY WARRANTY; without even the implied warranty of
% MERCHANTABILITY or FITNESS FOR A PARTICULAR PURPOSE.  See the
% GNU General Public License for more details.
% 
% You should have received a copy of the GNU General Public License
% along with this program.  If not, see <http://www.gnu.org/licenses/>.
%%

\anexo{Curriculum Vitae}

\anexo{Instrumento de Recolección de Datos}

\end{anexos}
	
	
\end{document}	
