%!TEX root = ../main.tex
\introduccion

El desarrollo de un sector de la robótica está dirigido hacia la creación de 
robots cada vez más semejantes al ser humano, tanto en términos corporales como en 
términos de inteligencia. Corporalmente, implica un diseño mecánico para que la 
estructura siempre permanezca en equilibrio. Este equilibrio será estable si la 
estructura mantiene el equilibrio para cualquier sistema de cargas aplicadas, ó 
inestable si se mantiene sólo para un conjunto particular. 

En términos de inteligencia es necesario modelar el raciocinio que tienen los 
seres vivos para reaccionar ante cualquier situación, lo cual siempre conlleva hacia 
una toma de decisiones y, la calidad de éstas, dependerá netamente de la herramienta 
utilizada para crear Inteligencia Artificial. 

 Existen diversas herramientas para tomar decisiones a nivel de inteligencia 
artificial, por ejemplo: las redes neuronales emplean algoritmos de aprendizaje 
iterativo, los sistemas inteligentes utilizan inferencias lógicas, los algoritmos 
genéticos aplican la teoría de la evolución darwiniana, entre otros. 

 La teoría de juegos también permite tomar decisiones. Aunque su principal 
campo de acción se ubica en la economía, ésta describe matemáticamente cualquier 
situación en la que dos o más individuos toman decisiones en búsqueda de un 
resultado que genere bienestar. Dicho bienestar puede ser individual o grupal, y 
dependerá netamente de la estrategia que utilice cada participante.  

 Un concepto fundamental en la teoría de juegos es el Equilibrio Nash. Este 
define una estrategia para cada individuo que participa en el “juego” y, cuando éstas 
son utilizadas, todos los participantes obtendrán el mejor bienestar posible. En otras 
palabras, se habrán tomado las decisiones óptimas. 

 Este trabajo de grado se basa en la construcción de un robot inteligente capaz 
de tomar decisiones óptimas en situaciones finitas de competencia que, a modo de 
ejemplo, están enfocadas al juego Tres en Línea. 

 Este trabajo de grado se estructura como sigue: el Capítulo I presenta la 
descripción del problema, justificación de la investigación, objetivos, alcances y 
limitaciones de la misma. El Capítulo II presenta los antecedentes de la investigación 
junto a las bases teóricas del trabajo, las cuales son divididas en secciones: Mecánica, 
Electrónica y Teoría de Juegos.  

 El Capítulo III describe el marco metodológico donde se encuentra la 
naturaleza del trabajo y las tres fases para el desarrollo de la investigación: 
Diagnóstico, Factibilidad y Diseño. Además, explica brevemente el diseño del 
sistema.  

 El Capítulo IV presenta los resultados obtenidos y se explica detalladamente 
los módulos que integran al sistema; en el Capítulo V se presentan las conclusiones y 
recomendaciones. 