\capitulo{Ambientes Principales}

\co{hacerbibliografia}\pa\ crea la bibliografía y la agrega al índice general. Usado después del ambiente \texttt{\textbf{contenido}} y antes del ambiente \texttt{\textbf{anexos}} si este se incluye.

\seccion{Preliminares}

\co{hacercaratula} crea la caratula exterior de la tesis.

\co{hacerpresentacion} crea la hoja de presentación de la tesis.

\co{haceraprobacion} crea la hoja de aprobación requerida para el trabajo final.

\co{preliminar}\pa\ para tilular e indexar un capitulo de los preliminares.

\co{hacerindices} crea todos los indices.

\co{hacerresumen} crea la hoja con el resumen de la tesis.

\co{hacerabstract} crea la hoja con el abstract de la tesis (opcional).


\seccion{Contenido}

\co{introduccion} usado al principio de la introducción para indexar la misma y agregar la palabra ``CAPÍTULO'' al índice general.

\co{capitulo}\pa\ para tilular e indexar un capitulo del contenido.

Comandos \co{seccion}\pa, \co{subseccion}\pa\ y \co{subsubseccion}\pa, para agregar e indexar títulos de segundo, tercero y cuarto nivel.


\seccion{Anexos}

\co{anexo}\pa\ para tilular e indexar un anexo.

