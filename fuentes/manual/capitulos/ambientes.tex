\capitulo{Ambientes Indexados y Especiales}

Se puede hacer referencia a los ambientes con etiquetas mediante el uso del comando \co{ref}\pa[etiqueta].

\seccion{Ilustración}

\am{ilustracion}{ para mostrar una ilustración. Los parámetros opcionales la agregan al índice de ilustraciones}{[\pa[título]\pa[etiqueta]]}{\co{includegraphics}\pa[ilustraciones/art.jpg]}
Se puede hacer referencia a una ilustracion usando el comando \co{refilustracion}.
Si se desea agregar una fuente a una ilustracion se puede usar el comando: \\ \co{tituloyfuenteilustracion}, ejemplo:

\begin{listado}{[LaTeX]TeX}
  \begin{ilustracion}
      \includegraphics{ruta_archivo}
      \tituloyfuenteilustracion{título}{fuente}{etiqueta}
  \end{ilustracion}
\end{listado}

\seccion{Gráfico}

\am{grafico}{ para mostrar un gráfico. Los parámetros opcionales lo agregan al índice de gráficos}{[\pa[título]\pa[etiqueta]]}{\co{includegraphics}\pa[graficos/fitness.pdf]}

Se puede hacer referencia a un grafico usando el comando \co{refgrafico}.
Si se desea agregar una fuente se puede usar el comando: \\ \co{tituloyfuentegrafico}, ejemplo:

\begin{listado}{[LaTeX]TeX}
  \begin{grafico}
      \includegraphics{ruta_archivo}
      \tituloyfuentegrafico{título}{fuente}{etiqueta}
  \end{grafico}
\end{listado}

\seccion{Cuadro}

\am{cuadro}{ para mostrar una tabla y agregarla al índice de cuadros}
{[tamáño de letra]\\\pa[título en el cuadro]\pa[etiqueta]\pa[columnas]}{\ldots}

\am{cuadroconindice}{ para mostrar una tabla y agregarla al índice de cuadros, con un índice personalizado}
{[tamáño de letra]\\\pa[título en el índice]\pa[título en el cuadro]\pa[etiqueta]\pa[columnas]}{\ldots}

Utilizar el comando \co{fuentecuadro}\pa[\# columnas en el cuadro]\pa[fuente]\ dentro del ambiente para agregar la fuente de donde se tomó el cuadro.

Véase el uso del ambiente \texttt{\textbf{tabular}} o del paquete \textbf{ctable}; o véase el código fuente del siguiente ejemplo.

Se puede hacer referencia a un cuadro usando el comando \co{refcuadro}.

Referencia rápida en \url{http://en.wikibooks.org/wiki/LaTeX/Tables}.


\begin{cuadro}
{Ejemplo de Cuadro: Piedra, Papel ó Tijeras – Forma Normal}
{tab:ejemplo}
{lccc}
	\toprule
	Jugares I/II & Piedra & Papel & Tijeras\\
	\midrule
	Piedra   & (0,0) & (-1,1) & (1,-1)\\
	Papel   & (1,-1) & (0,0) & (-1,1)\\
	Tijeras   & (-1,1) & (1,-1) & (0,0)\\
	\bottomrule
	\fuentecuadro{4}{Juan Rada (2005)}
\end{cuadro}

\begin{cuadroconindice}
{Ejemplo de Cuadro Con Índice}
{Piedra, Papel ó Tijeras – Forma Normal}
{tab:ejemplo2}
{lccc}
	\toprule
	Jugares I/II & Piedra & Papel & Tijeras\\
	\midrule
	Piedra   & (0,0) & (-1,1) & (1,-1)\\
	Papel   & (1,-1) & (0,0) & (-1,1)\\
	Tijeras   & (-1,1) & (1,-1) & (0,0)\\
	\bottomrule
	\fuentecuadro{4}{Juan Rada (2005)}
\end{cuadro}

\pagebreak

\seccion{Cita textual}

\am{citatextual}{ Prepara el contexto para una cita textual de más de 40 palabras}
{}{...Esto es una cita textual... (p. 199)}


\seccion{Ecuación}

\am{ecuacion}{ para insertar una ecuación con etiqueta para hacerle referencia}
{\pa[etiqueta]}{y = x + 1}

\espaciodoble

\am{ecuaciones}{ para insertar una lista de ecuaciones numeradas}
{}{y = x + 1 \textbackslash\textbackslash\\ z = x + 2}

\seccion{Listado}

\am{listado}{ para insertar un listado, usado principalmente para códigos y algoritmos, en el parámetro puede indicarse un lenguaje ya soportado, uno definido por el usuario o nada}
{\pa[lenguaje]}{\ldots}

\am{listadoindexado}{ funciona igual que \texttt{\textbf{listado}}, incluyendo un título y agregándolo al índice de listados}
{\pa[título]\pa[etiqueta]\pa[lenguaje]}{\ldots}

\am{listadoindexadoconfuente}{ funciona igual que \texttt{\textbf{listadoindexado}}, incluyendo un título, fuente y agregándolo al índice de listados}
{\pa[título]\pa[fuente]\pa[etiqueta]\pa[lenguaje]}{\ldots}

Se puede hacer referencia a un grafico usando el comando \co{reflistado}.

Más información en \url{http://en.wikibooks.org/wiki/LaTeX/Source_Code_Listings}.

Utilizar el comando \co{letralistados} para restablecer la letra por defecto, utilizar con un parámetro opcional (entre \texttt{[]}) para cambiar el estilo de la letra.

Utilizar el comando \co{lstdefinelanguage}\pa\pa\ y la clave \texttt{morekeywords} del paquete \textbf{listings} para definir un lenguaje personalizado. Véase la documentación.

Utilizar el comando \co{definirliterales}\pa\ para definir literales de reemplazo. Los parámetros tienen la forma \texttt{\pa[match]\pa[\pa[reemplazo]]largo} y van separados sólo por espacios. Véase la documentación. Ejemplo:

\definirliterales{{~}{{}}0}

\begin{listado}{[LaTeX]TeX}
\definirliterales{
	{:=}{{$\leftarrow$}}2
	{<<}{{$\langle$}}1 {>>}{{$\rangle$}}1
}
\begin{listado~}{}
 x := <<valor>>
\end{listado~}
\end{listado}

\vspace{-16pt}\noindent
resulta en:
\vspace{-12pt}

\definirliterales{{:=}{{$\leftarrow$}}2 {<<}{{$\langle$}}1 {>>}{{$\rangle$}}1}

\begin{listado}{}
 x := <<valor>>
\end{listado}

\seccion{Algoritmo}

\am{algoritmo}{ para escribir algoritmos en pseudocodigo. Los parámetros opcionales lo agregan al índice de algoritmos}{[\pa[título]\pa[etiqueta]]}{\dots}

Se puede hacer referencia a un algoritmo usando el comando \co{refalgoritmo}.

Este ambiente usa el paquete \textbf{algorithm2e} \url{http://en.wikibooks.org/wiki/LaTeX/Algorithms#Typesetting_using_the_algorithm2e_package}.

Si se desea agregar una fuente a un algoritmo se puede usar el comando: \\ \co{tituloyfuentealgoritmo}, ejemplo:

\begin{listado}{[LaTeX]TeX}
  \begin{algoritmo}
      ...
      \tituloyfuentealgoritmo{título}{fuente}{etiqueta}
  \end{algoritmo}
\end{listado}

\begin{algoritmo}
 \KwData{this text}
 \KwResult{how to write algorithm with \LaTeX2e }
 initialization\;
 \While{not at end of this document}{
  read current\;
  \eIf{understand}{
   go to next section\;
   current section becomes this one\;
   }{
   go back to the beginning of current section\;
  }
 }
 \tituloyfuentealgoritmo{Ejemple de algoritmo con fuente}{\cite{wikialg}}{algoritmo-ejemplo}
\end{algoritmo}
