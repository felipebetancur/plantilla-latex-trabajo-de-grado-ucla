\capitulo{Comandos Misceláneos}

\seccion{Pendientes}

\co{pendiente}\pa\ permite anotar un pendiente.

\co{habilitarpendientes}\ muestra los pendientes y el índice de pendientes.

\seccion{Notas}

\co{nota}\pa\ permite crear una anotación.

\co{habilitarnotas}\ muestra las anotaciones y el índice de anotaciones.

\seccion{Espaciado}

\co{espaciodoble}\ deja espacio vertical de 12pt.

\co{espaciotriple}\ deja espacio vertical de 22pt.

\co{tab}\pa[\#]\ deja espacio horizontal de \texttt{\#}em.

\co{tabm}\ deja un espacio horizontal en modo math.

\co{pagenblanco} para insertar una página completamente en blanco (sin enumeración).

\seccion{Otros}

\co{yo}\ muestra la referencia al autor del trabajo de grado, por ejemplo: \yo. Para usar este comando previamente se debe
se debe asignar el apellido del autor con el comando \co{citarcomo}\ en la sección de identificación, ejemplo: \co{citarcomo\{Piña\}}.

\co{comillas}\ coloca un texto entre comillas dobles, ejemplo: \comillas{Ejemplo}.
