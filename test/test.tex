\documentclass{uclamsc}

% Archivo para pruebas

\begin{document}

\begin{preliminares}	
	\hacerindices
\end{preliminares}

\begin{contenido}

Este documento se generó para probar la clase \comillas{uclamsc}

\refalgoritmo{alg:construccion-de-nido-de-avispas}

\begin{algoritmo}
\footnotesize
Inicialización\;
Construir la tabla de consulta, igual para todos los agentes\;
Colocar el primer bloque en un sitio predefinido\;
\For{$k \leftarrow 1$ \KwTo $m$}{
    Asignar agente $k$ aleatoriamente en sitio desocupado\;
}
\For{$t \leftarrow 1$ \KwTo $t\_maximo$}{
    \For{$k \leftarrow 1$ \KwTo $m$}{
	Evaluar la configuración local del agente $k$\;
	\eIf{La configuración local está en la tabla de consulta}{
	    Depositar el bloque especificado en la tabla de consulta\;
	}{
	    No depositar el bloque\;
	}
		Moverse agente $k$ hasta seleccionar aleatoriamente un sitio vecino y desocupado\;
    }
}
\tituloyfuentealgoritmo{Algoritmo de construcción de nido de avispas.}{Esta es la fuente}{alg:construccion-de-nido-de-avispas}
\end{algoritmo}

\refalgoritmo{alg:construccion-de-nido-de-avispas2}

\begin{algoritmo}
\footnotesize
Inicialización\;
Construir la tabla de consulta, igual para todos los agentes\;
Colocar el primer bloque en un sitio predefinido\;
\For{$k \leftarrow 1$ \KwTo $m$}{
    Asignar agente $k$ aleatoriamente en sitio desocupado\;
}
\tituloyfuentealgoritmo{Algoritmo de construcción de nido de avispas.}{Esta es la fuente}{alg:construccion-de-nido-de-avispas2}
\end{algoritmo}

\begin{listado}{[LaTeX]TeX}
  \begin{ilustracion}
      \includegraphics{ruta_archivo}
      \tituloyfuenteilustracion{título}{fuente}{etiqueta}
  \end{ilustracion}
\end{listado}

\end{contenido}

%\define@key{test}{ppp}{\def\test@ppp{#1}}
%\define@key{test}{ppa}{\def\test@ppa{#1}}
%% \setkeys{test}{ppp=yyy}
%  
%\newcommand{\test}[2][]{
%  \setkeys{test}{#1}
%  ppp: \ifthenelse{\isundefined{\test@ppp}}{ppp vacio}{\test@ppp}; \par
%  ppa: \ifthenelse{\isundefined{\test@ppa}}{ppa vacio}{\test@ppa}; \par
%  input: #2;  
%}

\end{document}	