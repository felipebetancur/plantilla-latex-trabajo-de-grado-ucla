\introduccion

%Manual de comandos del documentclass uclamsc, para los trabajos de grado de la Universidad Centroccidental Lisandro Alvarado (UCLA), Venezuela.

\noindent
Página del proyecto: \\ {\footnotesize \url{https://bitbucket.org/sauljabin/trabajo-de-grado-ucla-plantilla-latex}}

\noindent
Esta versión es derivada de la versión 2.0: \\ {\footnotesize \url{https://code.google.com/p/uclamsc}}

El código fuente para la creación de este documento (\texttt{manual.pdf}), se encuentra en el directorio del proyecto \texttt{manual}.

En el directorio \texttt{trabajo} se encuentra la plantilla vacía para el trabajo de grado, que se puede usar como punto de partida.

El cuerpo básico para el documento tiene la siguiente forma:

\lstset{commentstyle=\color[gray]{0.3}\itshape, escapeinside={(*}{*)}}

\begin{listado}[titulo=Cuerpo del Documento, etiqueta=lst:body]{[LaTeX]TeX}
\documentclass{uclamsc}
\includeonly{
	Capítulos a incluir en la compilación
}
\bibliografia{(*\textit{nombre del archivo principal}*)}
\habilitarpendientes	
\habilitarnotas
Comandos de indentificación
\begin{document}
	Incluir glosario si se va a usar %!TEX root = ../trabajo.tex 

\capitulo{Glosario}

\co{hacerglosario}\ ajusta y titula ``Definición de Términos Básicos'' la sección de términos del glosario insertada por el comando \co{printglossary}. Véase uso del paquete \textbf{glossaries} (\url{http://www.ctan.org/tex-archive/macros/latex/contrib/glossaries}).

Referencia rápida en \url{http://en.wikibooks.org/wiki/LaTeX/Glossary}.

\co{hacerglosarioconacronimos}, igual que \co{hacerglosario}\ pero titulando la sección ``Glosario de Acrónimos y Términos'', más apropiado para cuando se definen acrónimos en el glosario.

Para agregar un término al glosario se debe usar el comando \co{agregartermino}, 
para referirse al término usar el comando \co{gls}. Los términos deben ser agregados en el archivo \textbf{capitulos/glosario}, y debe incluirse en el cuerpo
del documento con el comando \co{input}\pa[capitulos/glosario].

A continuación un ejemplo de como agregar un término.

\begin{listado}[titulo=Ejemplo de término para el glosario]{[LaTeX]TeX}
\agregartermino{latex}{
  name={Latex}, 
  text={Latex},
  description={
    Es un sistema de composición de textos, 
    orientado especialmente a la creación de libros,
    documentos científicos y técnicos que contengan 
    fórmulas matemáticas
  }
}
\end{listado}

A continuación se muestra el resultado del comando \co{hacerglosario}. Se invoca el termino \gls{latex}.

\hacerglosario	
	Incluir resumen \resumen{}
	Incluir abstract \abstract{}
	\begin{preliminares}
		Comandos de preliminares
	\end{preliminares}
	\begin{contenido}
		Comandos de contenido
	\end{contenido}
	\hacerbibliografia
	\begin{anexos}
		Comandos de anexos
	\end{anexos}
\end{document}
\end{listado}