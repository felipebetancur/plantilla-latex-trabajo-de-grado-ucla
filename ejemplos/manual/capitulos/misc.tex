\capitulo{Comandos Misceláneos}

\co{pendiente}\pa\ permite anotar un pendiente.

\co{habilitarpendientes}\ muestra los pendientes y el índice de pendientes.


\co{nota}\pa\ permite crear una anotación.

\co{habilitarnotas}\ muestra las anotaciones y el índice de anotaciones.


\seccion{Glosario}

\co{hacerglosario}\ ajusta y titula ``Definición de Términos Básicos'' la sección de términos del glosario insertada por el comando \co{printglossary}. Véase uso del paquete \textbf{glossaries} (\url{http://www.ctan.org/tex-archive/macros/latex/contrib/glossaries}).

Referencia rápida en \url{http://en.wikibooks.org/wiki/LaTeX/Glossary}.

\co{hacerglosarioconacronimos}, igual que \co{hacerglosario}\ pero titulando la sección ``Glosario de Acrónimos y Términos'', más apropiado para cuando se definen acrónimos en el glosario.


\seccion{Espaciado}

\co{espaciodoble}\ deja espacio vertical de 12pt.

\co{espaciotriple}\ deja espacio vertical de 22pt.

\co{tab}\pa[\#]\ deja espacio horizontal de \texttt{\#}em.

\co{tabm}\ deja un espacio horizontal en modo math.

\co{pagenblanco} para insertar una página completamente en blanco (sin enumeración).



